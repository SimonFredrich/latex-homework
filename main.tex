% Todas as linhas precedidas pelo simbolo '%' são comentários
% e não afetam em nada o seu texto final.

% IGNORE. Pacotes necessários e acessórios para o documento
\documentclass[12pt, a4paper]{article}
\usepackage{amsthm}
\usepackage{libertine}
\usepackage[utf8]{inputenc}
\usepackage[margin=1in]{geometry}
\usepackage{amsmath,amssymb}
\usepackage{multicol}
\usepackage[shortlabels]{enumitem}
%--------------------------------------
% Informações que podem ser configuradas
%--------------------------------------
\newcommand{\class}{Romantik}
\newcommand{\term}{3. Semester}
\newcommand{\topic}{Einfluss der französischen Revolution auf Deutschland}
\newcommand{\created}{21/08/2020}
\newcommand{\timelimit}{}
\newcommand{\name}{Simon Fredrich}
%--------------------------------------
%encoding
%--------------------------------------
\usepackage[utf8]{inputenc}
\usepackage[T1]{fontenc}
%--------------------------------------
%German-specific commands
%--------------------------------------
\usepackage[ngerman]{babel}
%--------------------------------------
%Hyphenation rules
%--------------------------------------
\usepackage{hyphenat}
\hyphenation{Mathe-matik wieder-gewinnen}
%--------------------------------------
% change size of headings
%--------------------------------------
\usepackage[small]{titlesec}
%--------------------------------------
% change text-offset settings
%--------------------------------------
\addtolength{\voffset}{-2cm}
\addtolength{\hoffset}{-0.25cm}
\addtolength{\textwidth}{2cm}

\usepackage{fancyhdr}

\begin{document}
%\pagestyle{plain}
%\thispagestyle{empty}
% formalities
%--------------------------------------
\noindent
\begin{tabular*}{\textwidth}{l @{\extracolsep{\fill}} r @{\extracolsep{6pt}} l}
\textit{\name} && \textit{\created}\\             % Insira o seu nome dentro dos {}'.
\textit{\class} &&\textit{\term}\\
\end{tabular*}\\
% big line to separate formalities
\rule[1ex]{\textwidth}{0.5pt}
% topic heading
%\begin{center}{\scshape \Large \topic}\end{center}
\begin{center}{\Large \topic}\end{center}
%--------------------------------------
% set pagenumbering
%--------------------------------------
%\setcounter{page}{1}
\pagenumbering{arabic}
\rfoot[]{\thepage}
\section{Erstens}
Beispiel I.

\bigskip

a) $A \lor B, \neg A \vDash B$

\begin{proof}
Iremos demonstrar que o presente argumento é válido. Suponha, por absurdo, que o argumento é inválido. Assim, há uma valoração $v$, tal que:
i. $v(A\lor B)=V$, 
ii. $v(\neg A)=V$ e 
iii. $v(B)=F$. Note que de i. e iii., pelo significado da ($\lor$), temos que iv. $v(A)=V$. De iv., pelo significado da ($\neg$), temos que v. $v(\neg A)=F$. Contudo, de ii. e v., obtemos uma contradição, visto que $v$ é função. Segue-se disso que não há valoração que torne as premissas verdadeiras e a conclusão falsa. Portanto, o argumento é válido.\\
\end{proof}

\bigskip

\end{document}
